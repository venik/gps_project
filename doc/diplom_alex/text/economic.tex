\section{ОРГАНИЗАЦИОННО-ЭКОНОМИЧЕСКИЙ РАЗДЕЛ}
\subsection{Планирование разработки программных средств с построением графика}
Целью дипломного проекта является разработка программного комплекса (ПК) оценки трудоемкости объектно-ориентированных программ.
В данном разделе определяется трудоёмкость и затраты на создание ПК, а так же производится расчёт основных технико-экономических
показателей проекта.

\subsubsection*{Определение трудоемкости и продолжительности работ по созданию ПК}
Процесс разработки включает: обзор и анализ программных средств схожей тематики, анализ и выбор программных продуктов для
создания программы; отладка; испытание. В свою очередь каждый из этих этапов можно подразделить на отдельные под этапы.
Согласно ГОСТ 23501.1-79 регламентируются следующие стадии проведения исследования:

\begin{itemize}
	\item техническое задание – ТЗ (ГОСТ 23501.2-79);
	\item эскизный проект – ЭП (ГОСТ 23501.5-80);
	\item технический проект – ТП (ГОСТ 23501.6-80);
	\item рабочий проект – РП (ГОСТ 23501.11-81);
	\item внедрение – ВП (ГОСТ 23501.15-81).
\end{itemize}

Планирование стадий и содержания работ осуществляется в соответствии с \cite{bibl51}. На всех стадиях проведения исследования
выполняются следующие виды работ, перечень которых показан в таблице ~\ref{tab:eco1}.

% http://users.sdsc.edu/~ssmallen/latex/longtable.html
\begin{center}
\begin{longtable}{|l|l|}
\caption{Состав работ и стадии разработки ПК} \label{tab:eco1} \\ \hline
\multicolumn{1}{|c|}{\textbf{Стадии разработки}}    &   \multicolumn{1}{c|}{\textbf{Перечень работ}} \\ \hline
\multicolumn{1}{|c|}{\textbf{1}}    &   \multicolumn{1}{c|}{\textbf{2}} \\ \hline
\endfirsthead

%\multicolumn{2}{c} %
%{{\bfseries \tablename \thetable{} -- continued from previous page}} \\
\multicolumn{2}{|l|}{{Продолжение таблицы ~\ref{tab:eco1}}} \\ %\hline
\hline \multicolumn{1}{|c|}{\textbf{1}} &
%\multicolumn{1}{c|}{\textbf{Triple chosen}} &
\multicolumn{1}{c|}{\textbf{2}} \\ \hline 
\endhead

%\multicolumn{2}{|r|}{{}} \\ %\hline
%\hline \multicolumn{2}{|r|}{{Continued on next page}} \\ %\hline
\endfoot

\hline
\endlastfoot
		Техническое задание & \begin{parbox}{5in} {
						\begin{itemize}
							\item постановка задачи;
							\item подбор литературы;
							\item сбор исходных данных;
							\item определение требований к системе;
							\item определение стадий, этапов и сроков разработки ПК;
						\end{itemize} }  
					\end{parbox} \\
	\hline
		Эскизный проект & \begin{parbox}{5in} {
					\begin{itemize} 
						\item анализ программных средств схожей тематики;
						\item разработка общей структуры ПК;
						\item разработка структуры программы по подсистемам;
						\item документирование;
					\end{itemize} }
				\end{parbox} \\
	\hline
		Технический проект & \begin{parbox}{5in} {
					\begin{itemize} 
						\item определение требований к ПК;
						\item выбор инструментальных средств;
						\item определение свойств и требований к аппаратному обеспечению;
					\end{itemize} } 
				\end{parbox} \\
	\hline
		Рабочий проект & \begin{parbox}{5in} {
					\begin{itemize} 
						\item программирование;
						\item тестирование и отладка ПК;
						\item разработка программной документации;
						\item согласование и утверждение программы и методики испытаний;
					\end{itemize} }
				\end{parbox} \\
	\hline
		Внедрение & \begin{parbox}{5in} {
					\begin{itemize}
						\item ;
						\item ;
						\item ;
						\item ;
					\end{itemize} }
				\end{parbox} \\
\end{longtable}
\end{center}

Трудоемкость выполнения работ по созданию ПК  на каждой из стадий определяется в соответствии с \cite{bibl52, bibl53}.

Трудоемкость разработки ПК определяется по сумме трудоемкости этапов и видов работ, оцениваемых экспертным путем в
человеко-днях, и носит вероятностный характер, так как зависит от множества трудно учитываемых факторов.

Трудоемкость каждого вида работ определяется в соответствии с методическими указаниями \cite{bibl53} по формуле:

\begin{equation}
t_i = \frac{3\cdot{t_{min}} + 2\cdot{t_{max}}}{5}
\label{eq:l1_freq}
\end{equation}

где:	$t_{min}$ - минимально возможная трудоемкость выполнения отдельного вида работ в человеко-днях; \\
	$t_{max}$ - максимально возможная трудоемкость выполнения отдельного вида работ в человеко-днях.

Продолжительность каждого вида работ в календарных днях $((T_i)$) определяется по формуле \cite{bibl53}:

\begin{equation}
T_i = \frac{t_i}{\mbox{Ч}_i}\cdot{K_{\mbox{вых}}}	% FIXME
\label{eq:l1_freq}
\end{equation}

\newpage

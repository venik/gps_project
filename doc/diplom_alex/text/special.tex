\section{СПЕЦИАЛЬНЫЙ РАЗДЕЛ}

%--------------------------------------------------------------------------------
\subsection{Математическая модель преобразований сигнала в приемнике}




%--------------------------------------------------------------------------------
\subsection{Технология точной оценки частоты несущей}
Так как всегда присутствуют сдвиги несущей С/A кода и GPS сигнала, C/A код должен быть быть извлечен из сигнала. Процесс сопровождения 
сигнала "следует" за сигналом и декодирует информацию из навигационных сообщений. Если GPS приемник стационарный, то ожидаемое
изменение частоты, обусловленное движением навигационного спутника, очень низкое (в \cite{tsui} рассчитаным значением для стационарного
приемника является 0.936 Гц/с, для приемника движущегося с ускорением свободного падения это значение является уже очень существенным - 51.5 Гц/с).
Учитывая это, локальное изменение частоты должно так же просиходить с малой скоростью, по этой причине скорость обновления в
локальном ФАПЧ может быть тоже малой. Таким образом, для слежения за GPS сигналом необходимо два ФАПЧ модуля. Один из модулей
используется для слежения за изменением несущей частоты, он называется несущей петлей (carrier loop). Другой модуль 
используется для слежения за С/A кодом, он называется кодовой петлей.

Наиболее используемыми в программной реализации GPS приемника являются два метода: ФАПЧ (PLL) и блочная подстройка синхронизирующегося
сигнала (block adjustment of synchronizing signall - BASS). BASS метод является немного чувствительным к шуму.


%--------------------------------------------------------------------------------
\subsection{Уточнение фазы расширяющего кода}

\newpage

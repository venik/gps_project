\section{СПЕЦИАЛЬНЫЙ РАЗДЕЛ}

%--------------------------------------------------------------------------------
\subsection{Математическая модель преобразований сигнала в приемнике}
GPS приемник состоит из RF-части, IF-части, и части обработки сигнала (аппаратно или программно ориентированной). RF (radio frequency)
состоит из всех компонентов до первого смесителя (рисунок \ref{pic:hw_receiver}). На приведенном рисунке в нее входят: антенна,
усилитель 1 (входит в состав активной антенны), bias-tee компонент (необходим для подачи питания с цифрового источника в аналоговый
кабель антенны для усилителя 1) и широкополосный фильтр. IF (intermidiate frequency) состоит из всех компонентов после первого
смесителя до АЦП или второго смесителя, в некоторых конструкциях приемников \cite{gps}.

\begin{figure}[H]
\begin{center}
	\scalebox{0.99}{\includegraphics[width=1\linewidth]{./pics/gps_receiver.eps}}
\end{center}
\caption{Аппаратная часть GNSS L1 приемника}
\label{pic:hw_receiver}
\end{figure}

Первым компонентом после антенны может быть усилитель (как на рисунке \ref{pic:hw_receiver}) или широкополосный фильтр. Оба
варианта имеют достоинства и недостатки.

Формула для вычисления коэффициента шума (noise figure), называется формулой Фрииса \cite{boyd} и может быть записана как:

\begin{equation}
F_{\mbox{системы}} = 
			F_1 + \frac{F_2 - 1}{G_1} + \frac{F_3 - 1}{G_1 G_2} + ... + \frac{F_N - 1}{G_1 G_2 ... G_{N1}},
		   =
			\frac{SNR_{in}}{SNR_{out}}
\label{eq:friis}
\end{equation}
где ${F_i}$ и ${G_i (i=1,2,...N)}$ - коэффициенты шума и усиления каждого компонента в RF цепи.

Если первым элементом является усилитель, то коэффициент шума приемника будет приблизительно равным коэффициенту шума
усилителя 1 (рисунок \ref{pic:hw_receiver}), который можеты быть менее 2 дБ. Влияние следующего элемента RF цепи, например 
широкополосного усилителя, на коэффициент шума будет снижен на коэффициент усиления усилителя 1 (рисунок \ref{pic:hw_receiver}).
Потенциальной проблемой данного подхода ялвяется то, что при сильном уровне сигнала полоса пропускания может достичь насыщения
и начать генерировать побочные частоты.

Основной задачей смесителя на рисунке \ref{pic:hw_receiver} является перевод сигнала с высокой (RF) частоты на более низкую (IF)
частоту. GPS L1 сигнал поступает на частоте 1575.42 МГц. Основной целью данной операции явлеяется перевод сигнала
в удобный для работы диапазон частот, в частности для подачи на АЦП. В дизайне, отображенном на рисунке \ref{pic:hw_receiver},
отражен один каскад понижения частоты, однако их может быть несколько \cite{gps}. 

% software gps, page 60
Смеситель работает в соответствии с тригонометрическим выражением:

\begin{equation}
\cos(\omega_1 t)\cos(\omega_2 t) = 
	\frac{1}{2}\cos((\omega_1 - \omega_2)t) + \frac{1}{2}\cos((\omega_1 + \omega_2)t)
\label{eq:muxer}
\end{equation}

В нашем случае ${\omega_1}$ соответствует GPS L1 RF частоте 1575.42 МГц, а требуемой частотой является IF частота 2.5 МГц. В соответствии 
с этими условиями локальная частота ${\omega_2}$ должны быть (1575.42 - 2.5) = 1572.92 МГц. Таким образом первая часть формулы
\ref{eq:muxer} будет IF частотой, а вторая будет отфильтрована фильтром 2 (рисунок \ref{pic:hw_receiver}). Формула \ref{eq:muxer} 
предствляет собой упрощенную математическую модель смесителя. В реальных условиях необходимо учитывать потери передачи, нелинейные
искажения, динамические диапазоны, затухание и т.д. В сложной модели необходимо учитывать множество факторов, например
фильтр 2 (рисунок \ref{pic:hw_receiver}) выбирается так, чтобы минимизировать нелинейные искажения, полученные на смесителе.


%--------------------------------------------------------------------------------
\subsection{Технология точной оценки частоты несущей}
Так как всегда присутствуют сдвиги несущей С/A кода и GPS сигнала, C/A код должен быть быть извлечен из сигнала. Процесс сопровождения 
сигнала "следует" за сигналом и декодирует информацию из навигационных сообщений. Если GPS приемник стационарный, то ожидаемое
изменение частоты, обусловленное движением навигационного спутника, очень низкое (в \cite{tsui} рассчитаным значением для стационарного
приемника является 0.936 Гц/с, для приемника движущегося с ускорением свободного падения это значение является уже очень существенным - 51.5 Гц/с).
Учитывая это, локальное изменение частоты должно так же просиходить с малой скоростью, по этой причине скорость обновления в
локальном ФАПЧ может быть тоже малой. Таким образом, для слежения за GPS сигналом необходимо два ФАПЧ модуля. Один из модулей
используется для слежения за изменением несущей частоты, он называется несущей петлей (carrier loop). Другой модуль 
используется для слежения за С/A кодом, он называется кодовой петлей.

Наиболее используемыми в программной реализации GPS приемника являются два метода: ФАПЧ (PLL) и блочная подстройка синхронизирующегося
сигнала (block adjustment of synchronizing signall - BASS). BASS метод является немного чувствительным к шуму.


%--------------------------------------------------------------------------------
\subsection{Уточнение фазы расширяющего кода}

\newpage

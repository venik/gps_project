\addcontentsline{toc}{section}{ВВЕДЕНИЕ}
\section*{ВВЕДЕНИЕ}
Целью данного дипломного проекта является разработка и реализация комплекса программно-аппаратных средств
для захвата и сопровождения сигнала спутниковой навигации. Оборудование захвата радиосигналов является очень востребованным.
На данный момент оно применяется в научных разработках посвещенных анализу и улучшению алгоритмов определения координат 
приемника сигналов, а также в таких специфических исследованиях, как определение погоды по уровню сигнала спутников.

Сложность подобных устройств определяет их достаточно высокую цену, а высокая коммерческая привлекательность решений в области
GNSS (Global navigation satellite system - система глобальной спутниковой навигации) систем способствует закрытию решений в
данной области. Вместе с тем существует достаточно устойчивый интерес к данной области,
о чем свидетельствуют ежегодные публикации книг по тематике в крупных издательских компаниях США. Создание платформы по захвату
GNSS данных позволит в реальных условиях поработать с алгоритмами захвата и сопровождения данных, а также с алгоритмами
декодирования GNSS сообщений.

Структура дипомного проекта включает в себя несколько разделов:
\begin{itemize}
	\item в исследовательском разделе рассмотрены современные тенденции в области захвата сигналов, приведены
	      аналоги от крупшейних мировых производителей;
	\item в специальном разделе рассмотрены решения в области сопровождение GPS сигнала, в также математическая модель GPS приемника; 
	\item в технологическом разделе рассмотрены разработанные решения и даны краткие описания модулей;
	\item в разделе безопасность жизнидеятельности рассмотрены рассмотрены факторы пожарной опасности, предложены мероприятия по
	      оснащению помещения в котором находится ВТ устройствами локального тушения пожаров, а также рассмотрена экологичность
	      производства ВТ;
	\item в организационно-экономическом разделе производится рассчет стоимости разработки программно-аппаратного комплекса.
\end{itemize}

\newpage
